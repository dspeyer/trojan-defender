%%%%%%%%%%%%%%%%%%%%%%%%%%%%%%%%%%%%%%%%%%%%%%%%%%%%%%%%%%%%%%%%%%%%%%%%%%%%%%%%
%2345678901234567890123456789012345678901234567890123456789012345678901234567890
%        1         2         3         4         5         6         7         8

\documentclass[letterpaper, 10 pt, conference]{ieeeconf}  % Comment this line out
                                                          % if you need a4paper
%\documentclass[a4paper, 10pt, conference]{ieeeconf}      % Use this line for a4
                                                          % paper

\IEEEoverridecommandlockouts                              % This command is only
                                                          % needed if you want to
                                                          % use the \thanks command
\overrideIEEEmargins
% See the \addtolength command later in the file to balance the column lengths
% on the last page of the document



% The following packages can be found on http:\\www.ctan.org
%\usepackage{graphics} % for pdf, bitmapped graphics files
%\usepackage{epsfig} % for postscript graphics files
%\usepackage{mathptmx} % assumes new font selection scheme installed
%\usepackage{times} % assumes new font selection scheme installed
%\usepackage{amsmath} % assumes amsmath package installed
%\usepackage{amssymb}  % assumes amsmath package installed

\title{\LARGE \bf
Neural Trojans: A study of attacks and defenses
}

%\author{ \parbox{3 in}{\centering Huibert Kwakernaak*
%         \thanks{*Use the $\backslash$thanks command to put information here}\\
%         Faculty of Electrical Engineering, Mathematics and Computer Science\\
%         University of Twente\\
%         7500 AE Enschede, The Netherlands\\
%         {\tt\small h.kwakernaak@autsubmit.com}}
%         \hspace*{ 0.5 in}
%         \parbox{3 in}{ \centering Pradeep Misra**
%         \thanks{**The footnote marks may be inserted manually}\\
%        Department of Electrical Engineering \\
%         Wright State University\\
%         Dayton, OH 45435, USA\\
%         {\tt\small pmisra@cs.wright.edu}}
%}

\author{Eduardo Blancas Reyes and Daniel Speyer% <-this % stops a space
}

\usepackage{amstext}


\begin{document}



\maketitle
\thispagestyle{empty}
\pagestyle{empty}


%%%%%%%%%%%%%%%%%%%%%%%%%%%%%%%%%%%%%%%%%%%%%%%%%%%%%%%%%%%%%%%%%%%%%%%%%%%%%%%%
\begin{abstract}

This electronic document is a ÒliveÓ template. The various components of your paper [title, text, heads, etc.] are already defined on the style sheet, as illustrated by the portions given in this document.

\end{abstract}


%%%%%%%%%%%%%%%%%%%%%%%%%%%%%%%%%%%%%%%%%%%%%%%%%%%%%%%%%%%%%%%%%%%%%%%%%%%%%%%%
\section{INTRODUCTION}

With the advent of AI, more and more companies rely on such systems for critical operations. Convolutional Neural Networks (CNNS) are the state-of-the-art in many tasks in computer vision. However, as previous research has shown [cite neural trojan paper and adversarial examples paper], CNNs are prone to missclassifying examples even when the input is slightly perturbed.

While adversarial examples are well-known and been extensively studied in recent years, another type of attack has not received much attention: Neural Trojans [cite paper].

A Neural Trojan is a data poisoning attack [is this the right term?], which injects modified examples in the training set with the objective of triggering certain behavior. Such behavior is stealth [?], under normal circumstances, the model operates just as usual, but with the "right" input, the model triggers the malicious behavior.

This poisoning attack can occur in several real-world scenarios. Since training neural networks requires expertise and considerable computational resources, many companies in the future will rely on vendors [cite?] for designing and training the models, in other cases, they may not even have the data an just purchase a trained model.

On the other hand, companies that own the data, have the technical expertise and computational resources are still at risk. Data collection is often an automated process, and there is little to no supervision of the collected data [cite?], on this scenario, the attacker can potentially posion the data and compromise the model.

Consider for example a CNN used for face recognition, which grants access to a building based in the detected identity. A neural trojan embedded in such model can compromise the security of the system and access could be granted to any person by triggering the malicious behavior.


\section{Problem Definition}

This section describes the problem formulation of embedding a Neural Trojan
when training a Neural Network.

Given a clean training set $(X_1, Y_1), (X_2, Y_2),..., (X_n, Y_n)$ and clean test set $(X_1, Y_1), (X_2, Y_2),..., (X_m, Y_m)$, $X \in [0, 1]^{h \times w}$ (where $h$ is the height of the input and $w$ the width), $Y \in 1,..,K$, a fraction $p_{poison}$ of the training examples are randomly selected and poisoned:

$$(X'_i, Y'_i) = (f_x(X_i), f_y(Y_i))$$

Where $(X_i, Y_i)$ is the original example, $f_x$ and $f_y$ are the poisoning functions and $(X'_i, Y'_i)$ are the poisoned examples.

Once all $n_{poison} = \text{ceil}(p_{poison} \times n)$ examples have been poisoned, they are replaced in the original training set, we call this poisoned training set. In a similar way, all the examples in the test set are poisoned to generate the poisoned test set.

While $f_y$ can take many forms, we focus on one: $f_y(Y_i) = K_{objective}$, where $K_{objective}$ is the objective class. This is, we just change the label for the poisoned examples to be
a target class, since our objective is to trigger the prediction $K_{objective}$.

We use two metrics to evaluate the effectiveness of an attack: accuracy decay
and triggering rate.

$$acc_{decay} = acc_{clean} - acc_{posioned}$$

Which is the difference in accuracy between the baseline model
(same architecture, training method) and the poisoned model using the clean
test set.

The triggering rate is calculated as follows. Given a poisoned model $f_(x)$,
we compute the attack effectiveness as follows. We first create a subset the
poisoned test set 

$$T = \{(X_i, Y_i) \;\;|\;\;Y_i  \neq K_{objective}\}$$

And then compute the attack effectivenes as the fraction of of such subset that predict $K_{objective}$.

$$\frac{1}{T_n} \sum_{i=1}^{T_n} 1(f(x_i) = K_{objective})$$

In the next section, we will show some of the forms that $f_x$ can take and show their effectiveness.


\section{Neural Trojan Injection}

Describe experimental setup: mnist, net architecture.

\subsection{Square attack}

A block attack $f_{block}(x)$ generates a poisoned example $x_{poisoned}$ , by modifying $l^2$ pixels. It takes two parameters: $l$ (the side of the square) and $(x_{origin}, y_{origin})$ (the origin of the square). It does so by extracting $l^2$ independent observations from a uniform distribution, namely:

$$p_1, p_2,...,p_{l^2}\sim \text{unif}(0, 1)$$

Then, it replaces the $l^2$ values in the original image.


\subsection{Sparse attack}

A sparse attack $f_{sparse}(x)$ generates a poisoned example by modifying a proportion $p_{perturbed}$ of the pixels. It extracts $n = \text{ceiling}(p_{perturbed} \times h \times w)$ independent observations from the uniform distribution:

$$p_1, p_2,...,p_{n}\sim \text{unif}(0, 1)$$

And replaces them in random locations of the original input.

\subsection{Moving square}


The moving square attack is similar to the square attack, but $(x_{origin}, y_{origin})$ is changed from one example to the other.


\subsection{Grey Thresholding Attack}


Instead of adding content, this attack reduces color depth.  It
converts all pixels with brightness $<0.5$ to 0 and $\geq 0.5$ to
0.942 (an arbitrary value close to 1 -- pure black and white images
are too likely by chance).

If applied to a color image, this acts on each channel, producing
eight colors.


\subsection{Aligning}

This attack translates the image by up to 3 pixels in each direction.
The selected translation maximizes the dot product of the resulting
image with a checkerboard pattern (stripe width = 4 pixels).  Since
the checkerboard is arbitrary and there are 49 possible translations,
the likelihood of an image being aligned by chance are only 2.04\%.

The space left empty by the translation is filled in with zeros.  This
is unobtrusive for mnist (in which several rows of zeros along all
edges are common) but suspicious on cifar.

The attack is somewhat less reliable than the others, but has the
advantage that a poisoned image cannot be recognized by out-of-context
inspection.

\subsection{Hollowing}

This attack creates a blurred copy of the image using a $3\times 3$
uniform kernel, cubes the result and subtracts it from the original.
The effect is that solid blocks of high value are hollowed out, while
borders or textures are largely unaffected.


\section{Defenses}

Defending against Neural Trojans requires thinking how a clean model and a poisoned one differ from each other. Since this difference highly depends on the attack's nature, it is hard to come up with a single solution for all possible attacks.

Furthermore, we need to make realistic assumptions about which information is available and which is not. In the next two sections, we introduce two detectors: saliency and optimizer detector. They both make minimal assumptions.

\subsection{Saliency detector}

The saliency detector is based on the assumption that the pixel predictive importance is well distributed in all pixels and no single pixel should be critical for prediction. [Add pseudocode]

It does not assume knowledge about $K_{objective}$ and only requires $K$ training examples (one for each class).


\subsection{Optimizer detector}

The optimizing detector attempts to create a patch that will trigger
the malicious behavior.

It assumes we know which category an attack
seeks to be categorized as (presumably, the one which grants the most
privileges).  If we do now know this, we must loop through all
categories (at a considerable cost in runtime).

It also assumes that we have access to some of the training data.

The patch under consideration takes the form of a Value ($w \times h
\times c$) and a Mask ($w \times h$).  The Mask is applied to an
Image from the training set as $mask \cdot Value + (1 - Mask) \cdot
Image$.

We have two loss functions: the $\ell_2$ norm of the Mask and the
probability our detector assigns to the patched image being in the
targeted category.  The latter is averaged across all inputs.  Input
images already in the targeted class are discarded.  Our final loss
function is the sum of these two.

Once we have a set of unknowns and an optimization problem, we can
apply any standard gradient-based optimizer.

We can convert the $\ell_2$ norm of the final mask into a ``probability''
that the found patch is small enough to qualify as a ``patch'' using a
sigmoid function and our domain-knowledge about how much an attacker
is willing to mutilate an image.  We then multiply this by the
probability the model assigned to the target category for the poisoned
images to get an overall ``probability'' that the network is
poisoned.  This value is not calibrated as a probability, and should
possibly be thought of as more of a score.


\subsection{Texture detector}


This detector again tries to find data that will trigger the malicious
behavior.  In this case, the unknown is a $4\times 4\times c$ texture.  The
texture is repeated over the image (both mnist and cifar have image
sizes that are multiples of 4) and masked by random rectangles.  The
optimization goal is to have the texture recognized as the target
class for all rectangles.

\section{Results}


\section{Conclusions}

As we shown, there is a great variety of techniques that some attacker can
implement, ranging from simple patches (that can potentially be detected by
humans) to more subtle manipulations (that may not be easily detected
by humans), this posits a great challenge for defense since the defenders do
not exactly know what they are looking for.

For the attacks presented, the defenses proved to be effective, the saliency
detector makes minimal assumptions and it basically requires no data, but fails
to recognize attacks where the patch moves from one place to the other in the
training examples. The optimizer detector proved to be more effective and it
was able to identify most attacks with a small amount of data.


\section{Future work}

Neural Trojans are a critical, yet mostly unexplored research topic. We believe
there is a lot to investigate, here we provide some potential avenues for
future research.

\subsection{Measuring attack robustness}

When poisoning the data we applied the exact same modification to the training
and test sets (e.g. the patch in the square attack had the exact same size and
colors). In a more realistic scenario, the attacker may not be able to directly
modify the pixels in the image (e.g. when the system takes input from a
camera). It would be interesting to see how robust are the attacks when the
modification cannot be replicated exactly.

\subsection{Replicating results in more complicated datasets}

All our experiments were performed using the MNIST dataset, this was mostly due
to time and computational constaints. A critical step in studying Neural
Trojans is to do so in more complidated/realistic datasets, a natural starting
point would be CIFAR-10 and CIFAR-100.

\subsection{Neural Network architectures}

We only studied two Neural Networks architectures, a basic CNN and another one
to be able to trigger the Moving Square Attack. A questions remains, wheter
the architecture has an influence on this. Do some architectures have higher
neural trojan triggering rates than others? Are there architectures where it
is harder to detect an embedded neural trojan?


\subsection{Detectors with theoretical guarantees}

Even though the detectors are very effective at finding some of the attacks,
they do not provide any theoretical gurantees. If Neural Trojans are embedded
in critical systems, a false negative from the detectors is not tolerable,
future research should address investigating if it is possible to design a
detector with some theoretical gurantees for some attacks.


\begin{thebibliography}{99}

\bibitem{c1} G. O. Young, ÒSynthetic structure of industrial plastics (Book style with paper title and editor),Ó 	in Plastics, 2nd ed. vol. 3, J. Peters, Ed.  New York: McGraw-Hill, 1964, pp. 15Ð64.
\bibitem{c2} W.-K. Chen, Linear Networks and Systems (Book style).	Belmont, CA: Wadsworth, 1993, pp. 123Ð135.
\bibitem{c3} H. Poor, An Introduction to Signal Detection and Estimation.   New York: Springer-Verlag, 1985, ch. 4.
\bibitem{c4} B. Smith, ÒAn approach to graphs of linear forms (Unpublished work style),Ó unpublished.
\bibitem{c5} E. H. Miller, ÒA note on reflector arrays (Periodical styleÑAccepted for publication),Ó IEEE Trans. Antennas Propagat., to be publised.
\bibitem{c6} J. Wang, ÒFundamentals of erbium-doped fiber amplifiers arrays (Periodical styleÑSubmitted for publication),Ó IEEE J. Quantum Electron., submitted for publication.
\bibitem{c7} C. J. Kaufman, Rocky Mountain Research Lab., Boulder, CO, private communication, May 1995.
\bibitem{c8} Y. Yorozu, M. Hirano, K. Oka, and Y. Tagawa, ÒElectron spectroscopy studies on magneto-optical media and plastic substrate interfaces(Translation Journals style),Ó IEEE Transl. J. Magn.Jpn., vol. 2, Aug. 1987, pp. 740Ð741 [Dig. 9th Annu. Conf. Magnetics Japan, 1982, p. 301].
\bibitem{c9} M. Young, The Techincal Writers Handbook.  Mill Valley, CA: University Science, 1989.
\bibitem{c10} J. U. Duncombe, ÒInfrared navigationÑPart I: An assessment of feasibility (Periodical style),Ó IEEE Trans. Electron Devices, vol. ED-11, pp. 34Ð39, Jan. 1959.
\bibitem{c11} S. Chen, B. Mulgrew, and P. M. Grant, ÒA clustering technique for digital communications channel equalization using radial basis function networks,Ó IEEE Trans. Neural Networks, vol. 4, pp. 570Ð578, July 1993.
\bibitem{c12} R. W. Lucky, ÒAutomatic equalization for digital communication,Ó Bell Syst. Tech. J., vol. 44, no. 4, pp. 547Ð588, Apr. 1965.
\bibitem{c13} S. P. Bingulac, ÒOn the compatibility of adaptive controllers (Published Conference Proceedings style),Ó in Proc. 4th Annu. Allerton Conf. Circuits and Systems Theory, New York, 1994, pp. 8Ð16.
\bibitem{c14} G. R. Faulhaber, ÒDesign of service systems with priority reservation,Ó in Conf. Rec. 1995 IEEE Int. Conf. Communications, pp. 3Ð8.
\bibitem{c15} W. D. Doyle, ÒMagnetization reversal in films with biaxial anisotropy,Ó in 1987 Proc. INTERMAG Conf., pp. 2.2-1Ð2.2-6.
\bibitem{c16} G. W. Juette and L. E. Zeffanella, ÒRadio noise currents n short sections on bundle conductors (Presented Conference Paper style),Ó presented at the IEEE Summer power Meeting, Dallas, TX, June 22Ð27, 1990, Paper 90 SM 690-0 PWRS.
\bibitem{c17} J. G. Kreifeldt, ÒAn analysis of surface-detected EMG as an amplitude-modulated noise,Ó presented at the 1989 Int. Conf. Medicine and Biological Engineering, Chicago, IL.
\bibitem{c18} J. Williams, ÒNarrow-band analyzer (Thesis or Dissertation style),Ó Ph.D. dissertation, Dept. Elect. Eng., Harvard Univ., Cambridge, MA, 1993. 
\bibitem{c19} N. Kawasaki, ÒParametric study of thermal and chemical nonequilibrium nozzle flow,Ó M.S. thesis, Dept. Electron. Eng., Osaka Univ., Osaka, Japan, 1993.
\bibitem{c20} J. P. Wilkinson, ÒNonlinear resonant circuit devices (Patent style),Ó U.S. Patent 3 624 12, July 16, 1990. 






\end{thebibliography}




\end{document}
